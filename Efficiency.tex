\documentclass{article}
\usepackage{mathtemplate}

\begin{document}

As an abuse of notation, write
\begin{equation*}
(a_1,\ldots,a_n) \in \mathrm{Data}
\end{equation*}
to mean that there is a row in $\mathrm{Data}$ such that when restricted to an understood set of attributes, the resulting tuple is $(a_1,\ldots,a_n)$.

Denote by
\begin{equation*}
O((a_1,\ldots,a_n),a) \coloneq (\text{number of occurrences of $(a_1,\ldots,a_n,a)$ in $\mathrm{Data}$})
\end{equation*}
where $a$ corresponds to an attribute not in the set of attributes understood to correspond to $(a_1,\ldots, a_n)$.

With this notation, the support of a functional dependency $F=\mathrm{Domain} \to \mathrm{Attribute}$ is given by
\begin{equation} \label{support definition}
\supp(F) = \dfrac{\sum_{(a_1,\ldots,a_n) \in \mathrm{Data}}{\max O( (a_1,\ldots, a_n), a)}}{\sum_{(a_1,\ldots, a_n) \in \mathrm{Data}}{\sum_a{O((a_1,\ldots, a_n),a)}}}
\end{equation}

We shall use the notation
\begin{equation*}
(a_1,\ldots, \hat{a_i},\ldots,a_n) = (a_1,\ldots, a_{i-1},a_{i+1},\ldots,a_n)
\end{equation*}
i.e. the hat indicates the non-presense of $a_i$ in the tuple.

\begin{lem}
There exists a finite collection of finite sets $A_{i,j}$ of non-negative reals such that
$\sum_j{\max_i{A_{i,j}}} \leq \max_i{\sum_j{A_{i,j}}}$.

There also exists a fintie collection of finite sets $A_{i,j}$ of non-negative reals such that
$\sum_j{\max_i{A_{i,j}}} \geq \max_i{\sum_j{A_{i,j}}}$.
\end{lem}
\begin{proof}
For the first claim, take
\begin{equation*}
\{ A_{1,1} = \{1,2\}, A_{1,2} = \{0,0\}\}
\end{equation*}
Then $\sum_j{\max_i{A_{i,j}}} = 2+0 = 2$ and $\max_i{\sum_j{A_{i,j}}} = \max(3,0) = 3$.

For the second claim, take 
\begin{equation*}
\{ A_{1,1}=\{1,1\}, A_{1,2} = \{2,0\}\}
\end{equation*}
Then $\sum_j{\max_i{A_{i,j}}} = 1+2 =3$ and $\max_i{\sum_j{A_{i,j}}} = \max(2,2)=2$.
\end{proof}

With this in mind, we quickly see that
\begin{equation*}
O((a_1,\ldots, \hat{a_i},\ldots, a_n),a) = \sum_{a_i}{O((a_1,\ldots, a_i,\ldots,a_n),a)}
\end{equation*}
As a result, we find that
\begin{align*}
\sum_{(a_1,\ldots, \hat{a_i},\ldots, a_n) \in\mathrm{Data}}{\sum_a{O((a_1,\ldots, \hat{a_i},\ldots, a_n),a)}} & = \sum_{(a_1,\ldots, \hat{a_i},\ldots, a_n) \in\mathrm{Data}}{\sum_{a_i}{\sum_a{O((a_1,\ldots,a_i,\ldots, a_n),a)}}} \\
& = \sum_{(a_1,\ldots, a_i,\ldots, a_n) \in\mathrm{Data}}{\sum_a{O((a_1,\ldots, a_i,\ldots, a_n),a)}}
\end{align*}

In other words, if $\mathrm{Domain}'$ is any subset of $\mathrm{Domain}$ and $F' = \mathrm{Domain}' \to \mathrm{Attribute}$, we see that the denominator of $\supp(F')$ in Equation~\ref{support definition} is the same as that for $\supp(F)$.

%Likewise, 
%\begin{align*}
%\sum_{(a_1,\ldots,\hat{a_i},\ldots, a_n) \in\mathrm{Data}}{\max_a{O( (a_1,\ldots, \hat{a_i},\ldots,a_n),a)}} & = \sum_{(a_1,\ldots, \hat{a_i},\ldots,a_n) \in \mathrm{Data}}{\max_a{\sum_{a_i}{O( (a_1,\ldots, a_i,\ldots,a_n),a)}}} \\
%& \leq \sum_{(a_1,\ldots, \hat{a_i},\ldots,a_n) \in \mathrm{Data}}{\sum_{a_i}{\max_a{O( (a_1,\ldots, a_i,\ldots,a_n),a)}}} \\
%& = \sum_{(a_1,\ldots, a_i,\ldots,a_n) \in \mathrm{Data}}{\max_a{O( (a_1,\ldots, a_i,\ldots,a_n),a)}}
%\end{align*}
%
%In other words, if $\mathrm{Domain}'$ is any subset of $\mathrm{Domain}$ and $F' = \mathrm{Domain}' \to \mathrm{Attribute}$, we see that the numerator of $\supp(F')$ in Equation~\ref{support definition} is the same as that for $\supp(F)$.
%
%Together,
%\begin{equation*}
%\supp(F') \leq \supp(F)
%\end{equation*}

\begin{tabular}[ c c c c ]
A & B & C & D \\
1 & 1 & 1 & 1 \\
1 & 2 & 1 & 1 \\
1 & 2 & 2 & 2 \\
2 & 1 & 1 & 2 \\
2 & 1 & 2 & 1
\end{tabular}




\end{document}